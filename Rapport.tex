\documentclass[a4paper, 12pt]{article}
\usepackage[latin1]{inputenc}
\usepackage[T1]{fontenc}
\usepackage[french]{babel}
\usepackage{graphicx}
\usepackage{amsmath}
\usepackage{amssymb}

\let\iff\Longleftrightarrow

\pagestyle{headings}

\title{Analyse \\
Devoir Maison 2}
\author{Quentin Januel}
\date{\today}

\begin{document}

\maketitle

\newpage

Pour tout $x \in \mathbb R$, on pose $f(x) = |x|^{\pi/4}\sin(x)$. \\ \\

\begin{enumerate}

\item{
% 1
$|x|^{\pi/4}$ n'a de sens que si $|x| \in \mathbb R^+$, ce qui ne pose donc pas de probl�me. Le reste est clairement d�fini pour tout $x \in \mathbb R$. \\

Sur $\mathbb R^*$, $f(x) = e^{\pi/4\ln|x|}\sin(x)$. Pouvant �tre d�compos�e en produits et compositions de fonctions continues, elle est donc continue sur $\mathbb R^*$. \\
Aussi,
\begin{gather*}
f(0) = 0 \\
\lim_{x \to 0}f(x) = 0
\end{gather*}
$f$ est donc �galement continue en $0$ et par cons�quent continue sur $\mathbb R$. \\

$f(-x) = |-x|^{\pi/4}\sin(-x) = -|x|^{\pi/4}\sin(x) = -f(x)$ \\
La fonction $f$ est donc impaire.
% 1
}
\\

\item{
% 2
Soit $(u_n)_{n \in \mathbb N^*} \in \mathbb R^{\mathbb N^*}$ une suite r�elle d�finie telle que \\$\forall n \in \mathbb N^*, u_n = f(\pi(n+1/2))$. \\

Posons ensuite
\begin{gather*}
a_n = u_{2n} = e^{\pi/4\ln n} \\
b_n = u_{2n+1} = -e^{\pi/4\ln n}
\end{gather*} \\
Les fonctions exponentielle et logarithme n�p�rien �tant toutes deux croissantes, on a donc \\
\begin{gather*}
\lim_{n \to +\infty} a_n = +\infty \\
\lim_{n \to +\infty} b_n = -\infty
\end{gather*}

Si une suite admet une limite $l$ \in \overline{\mathbb{R}}$, alors toute sous-suite de cette suite admet $l$ pour limite. \\
Par contrapos�e, puisque $(a_n)_{n \in \mathbb N}$ et $(b_n)_{n \in \mathbb N}$ sont deux sous-suites admettant deux limites diff�rentes, la suite $u_n$ n'admet pas de limite. \\

Ainsi on a
$$
\begin{cases}
	\lim_{n \to +\infty} u_n \notin \mathbb R \iff
	\forall l \in \mathbb R, \exists \varepsilon_l > 0, \forall N \in \mathbb N, \exists n > N, |u_n-l| > \varepsilon_l \\

	\lim_{n \to +\infty} u_n \neq +\infty \iff
	\exists A > 0, \forall N \in \mathbb N, \exists n > N, u_n < A \\

	\lim_{n \to +\infty} u_n \neq -\infty \iff
	\exists A < 0, \forall N \in \mathbb N, \exists n > N, u_n > A
\end{cases}
$$

Or, pour tout $n \in \mathbb N, n < \pi(n+1/2)$ et donc
$$
n > N \implies \pi(n+1/2) > N
$$ \\

En prennant $x = \pi(n+1/2)$, on a
$$
\begin{cases}
	\forall l \in \mathbb R, \exists \varepsilon_l > 0, \forall N \in \mathbb R, \exists x > N, |f(x)-l| > \varepsilon_l
	\iff \lim_{x \to +\infty} f(x) \notin \mathbb R \\

	\exists A > 0, \forall N \in \mathbb N, \exists x > N, f(x) < A
	\iff \lim_{x \to +\infty} f(x) \neq +\infty \\

	\exists A < 0, \forall N \in \mathbb N, \exists x > N, f(x) > A
	\iff \lim_{x \to +\infty} f(x) \neq -\infty
\end{cases}
$$
et ainsi, $f$ n'admet pas de limite en $+\infty$.
% 2
}
\\ \\

\item{
% 3
$$
f(x) =
\begin{cases}
	e^{\pi/4\ln(x)}\sin(x) & \text{si $x > 0$} \\
	e^{\pi/4\ln(-x)}\sin(x) & \text{si $x < 0$}
\end{cases}
$$

Pouvant �tre d�compos�e en produits et compositions de fonctions d�rivables sur $\mathbb R^*$, $f$ est donc d�rivable sur ce m�me intervalle.
\\
\begin{gather*}
\implies f'(x) =
\begin{cases}
	e^{\pi/4\ln(x)}\cos(x)+\frac{\pi/4}{x}e^{\pi/4\ln(x)}\sin(x) & \text{si $x > 0$} \\
	e^{\pi/4\ln(-x)}\cos(x)+\frac{\pi/4}{-x}e^{\pi/4\ln(-x)}\sin(x) & \text{si $x < 0$} \\
\end{cases}
\\
\iff \forall x \in \mathbb R^*, f'(x) = e^{\pi/4\ln|x|}(\cos(x)+\frac{\pi\sin x}{4|x|})
\end{gather*}
\\

\begin{align*}
f'(0) =& \lim_{h \to 0}\frac{f(h)-f(0)}{h} \\
=& \lim_{h \to 0}\frac{|h|^{\pi/4}\sin(h)-|0|^{\pi/4}\sin(0)}{h} \\
=& \lim_{h \to 0}\frac{|h|^{\pi/4}\sin(h)}{h} \\
=& \lim_{h \to 0}|h|^{\pi/4} \times \lim_{h \to 0}\frac{\sin(h)}{h} \text{(car les deux limites convergent)} \\
=& 0 \times 1 = 0
\end{align*}
\\

Au final, on a donc
$$
f'(x) =
\begin{cases}
	0 & \text{si $x = 0$} \\
	e^{\pi/4\ln|x|}(\cos(x)+\frac{\pi\sin x}{4|x|}) & \text{sinon}
\end{cases}
$$
% 3
}
\\ \\

\item{
% 4
Soit $g$ une fonction d�finie est born�e aux voisinage d'un point $a \in \mathbb R$.
$$
\lim_{x \to a}h(x) = 0 \iff \forall \varepsilon_0 > 0, \exists \mu_0 > 0, |x-a| < \mu_0 \implies |h(x)| < \varepsilon_0
$$
$$
\text{$g$ est born�e au voisinage de $a$} \iff \exists \mu_1 > 0, \exists B > 0, |x-a| < \mu_1 \implies |g(x)| < B
$$
\\

Soit $\varepsilon_1 > 0$, posons $\mu = \text{max}(\mu_0, \mu_1)$ et $\varepsilon_0 = \frac{\varepsilon_1}{B}$.
\\

On a alors :
$$
|x-a| < \mu \implies |x-a| < \mu_0 \implies |h(x)| < \varepsilon_0 = \frac{\varepsilon_1}{B}
$$
et
$$
|x-a| < \mu \implies |x-a| < \mu_1 \implies |g(x)| < B
$$
\\

Ainsi $|h(x)||g(x)| < \varepsilon_1 \iff |h(x)g(x)| < \varepsilon_1$ et donc
$$
\lim_{x \to a}h(x)g(x) = 0
$$
(en l'occurrence, puisque $x-a$ tend vers $0$ quand $x$ tend vers $a$, alors $\lim_{x \to a}(x-a)g(x) = 0$)
\\ \\

Soit $g(x) = \cos(x)+\frac{\pi\sin x}{4|x|}$ et $h(x) = e^{\pi/4\ln|x|}$ d�finies pour tout $x \in \mathbb R^*$ \\ \\

$g$ est born�e au voisinage de $0$ et $\lim_{x \to 0}h(x) = 0$. \\

On a donc 
$$
\lim_{x \to 0}g(x)h(x) = \lim_{x \to 0}f'(x) = 0
$$
Or $f'(0) = 0$ donc la fonction $f'$ est bien continue en $0$, ce qui prouve que la fonction $f$ est de classe $C^1$.
% 4
}
\\ \\

\item{
% 5
\begin{enumerate}
\item{
% a
$f$ est impaire donc sa d�riv�e est paire. \\
Puisque $\forall x \in \mathbb R, f'(x) = f'(-x)$, il est donc �videmment que si $f'(x) = 0$ alors $f'(-x) = 0$.
% a
}
\\ \\

\item{
% b
Sur l'intervalle $]0; \pi]$, $f'(x) = e^{\pi/4\ln x}(\cos x +\frac{\pi\sin x}{4x})$. \\
Puisque la fonction exponentielle ne s'annule jamais,
\begin{align*}
f'(x) = 0 \iff& \cos x +\frac{\pi\sin x}{4x} = 0 \\
\iff& \pi\sin x+4x\cos x = 0
\end{align*}
% b
}
\\ \\

\item{
% c
Posons $g \in \mathbb R^{]0; \pi]}$ telle que \\
\begin{align*}
g(x) =& \pi\sin x+4x\cos x \\
\implies g'(x) =& \pi\cos x+4\cos x-4x\sin x \\
=& (\pi+4)\cos x-4x\sin x
\end{align*}
Lorsque $x$ varie entre $\pi/2$ et $\pi$, $\cos x$ varie entre $0$ et $-1$ et est donc n�gatif, il en va donc de m�me pour $(\pi+4)\cos x$. \\
Quant � $\sin x$, il varie entre $1$ et $0$ et est donc positif, impliquant que $-4x\sin x$ est n�gatif. \\ \\
Ainsi, leur somme soit $g'(x)$ est n�gative sur l'intervalle $]\pi/2; \pi]$ et donc $g$ est d�croissante sur ce m�me intervalle. \\
Avec un raisonnement similaire, on en d�duit que $g$ est croissante sur $]0; \pi/2]$. \\

Or $g(\pi/2) = \pi > 0$ et $g(\pi) = -4\pi < 0$. \\
De plus $\lim_{x \to 0^+}g(x) = 0$. \\

D'apr�s le th�or�me des valeurs interm�diaires, l'�quation $g(x) = 0$ n'a pas de solution sur $]0; \pi/2]$ et poss�de une unique solution $x_0$ sur $]\pi/2; \pi]$. \\

L'�quation $\pi\sin x+4x\cos x = 0$ poss�de donc une unique solution $x_0$ sur $]0; \pi]$.
% c
}
\\ \\

\item{
% d
On sait d�j� que $g(\pi/2) > 0$. \\
Puisque $g(3\pi/4) = \frac{\pi}{\sqrt 2}-\frac{3\pi}{\sqrt 2} < 0$, on peut donc dire (toujours gr�ce au TVI) que $x_0$ se situe dans l'intervalle $]\pi/2; 3\pi/4[$.
% d
}
\\ \\

\item{
% e
Puisque $x_0 \in ]\pi/2; 3\pi/4[$, on sait que $z_0 \in tan(]\pi/2; 3\pi/4[-\pi/2)$
\begin{align*}
\implies z_0 \in [tan(\pi/2-\pi/2); tan(3\pi/4-\pi/2)] =& [tan(0); tan(\pi/4)] \\
=& [0; 1]
\end{align*} \\

Suite du 5. e)... ^^
% e
}
\\ \\

\item{
% f
blabla f
% f
}
\\ \\

\item{
% g
blabla g
% g
}
\\ \\

\item{
% h
blabla h
% h
}

\end{enumerate}
% 5
}
\\ \\

\item{
% 6
blabla 6
% 6
}


\end{enumerate}

\end{document}

